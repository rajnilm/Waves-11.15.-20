\let\negmedspace\undefined
\let\negthickspace\undefined
\documentclass[journal,12pt,twocolumn]{IEEEtran}
\usepackage{cite}
\usepackage{amsmath,amssymb,amsfonts,amsthm}
\usepackage{algorithmic}
\usepackage{graphicx}
\usepackage{textcomp}
\usepackage{xcolor}
\usepackage{txfonts}
\usepackage{listings}
\usepackage{enumitem}
\usepackage{mathtools}
\usepackage{gensymb}
\usepackage{comment}
\usepackage[breaklinks=true]{hyperref}
\usepackage{tkz-euclide} 
\usepackage{listings}
\usepackage{gvv}                                        
\def\inputGnumericTable{}                                 
\usepackage[latin1]{inputenc}                                
\usepackage{color}                                            
\usepackage{array}                                            
\usepackage{longtable}                                       
\usepackage{calc}                                             
\usepackage{multirow}                                         
\usepackage{hhline}                                           
\usepackage{ifthen}                                           
\usepackage{lscape}
\setlength{\arrayrulewidth}{0.5mm}
\setlength{\tabcolsep}{18pt}
\renewcommand{\arraystretch}{1.5}

\newtheorem{theorem}{Theorem}[section]
\newtheorem{problem}{Problem}
\newtheorem{proposition}{Proposition}[section]
\newtheorem{lemma}{Lemma}[section]
\newtheorem{corollary}[theorem]{Corollary}
\newtheorem{example}{Example}[section]
\newtheorem{definition}[problem]{Definition}
\newcommand{\BEQA}{\begin{eqnarray}}
\newcommand{\EEQA}{\end{eqnarray}}
\newcommand{\define}{\stackrel{\triangle}{=}}
\theoremstyle{remark}
\newtheorem{rem}{Remark}
\begin{document}


\title{Waves(20) 11.15}
\author{EE23BTECH11051-Rajnil Malviya}
\date{January 2024}



\maketitle

\subsection*{\textbf{Question :-}}
A train, standing at the outer signal of a railway station blows a whistle of frequency
400 Hz in still air. (i) What is the frequency of the whistle for a platform observer
when the train (a) approaches the platform with a speed of $10 ms^{-1} $, (b) recedes
from the platform with a speed of $10 ms^{-1} $? (ii) What is the speed of sound in each
case ? The speed of sound in still air can be taken as $340 ms^{-1} $.

\bigskip
Solution :-\\
         \begin{table}[h]
        
\begin{tabular}{ | m{3.4em} | m{5cm} | } 
  \hline
 \textbf{Symbol} & \textbf{Meaning of Symbol}  \\
 \hline
 $f$ & actual frequency of source  \\
\hline
$f_a'$ & frequency observed by observer when train is approaching observer  \\
\hline
 $f_r'$ &  frequency observed by observer when train is receding observer  \\
\hline
 $v$ & velocity of air in that medium  \\
\hline
$v_s$ & velocity of source which is train  \\
\hline
$v_o$ & velocity of observer  \\
\hline
\end{tabular}

    \end{table}

(i)  a. When the train approaches the platform (i.e., the observer at rest),
\bigskip
\begin{align}f'_a=f\times\frac{v}{v-v_s}\end{align}

$$f'_a=400\times\frac{340}{340-10}$$

$$f'_a=412.1212$$
\bigskip

b. When the train recedes the platform (i.e., the observer at rest),
\bigskip

$f'_r $ is frequency observed by observer when train is receding platform,\\
\begin{align}f'_r=f\times\frac{v}{v+v_s}\end{align}

$$f'_r=400\times\frac{340}{340+10}$$

$$f'_r=388.5714$$\\
(ii) The speed of sound in each will be same.It is $340  ms^{-1}$ in each case.
\subsection*{\textbf{Equation of Sound Wave :-}}
Sound Wave is transmission of energy ; sound wave depends on many parameters . A general equation of sound wave is shown below 
\begin{align}y(t) = Asin( 2 \pi ft + \phi ) \end{align} 
\textit{y(t) is instantaneous 
displacement of wave at time t;}$$\textit{A is amplitude of wave;}$$$$f\;is\; frequency\; of\; wave;$$
$$t \;is\; time;$$$$\phi \; is \; phase \; angle;$$
\includegraphics[width=0.45\textwidth]{figs/waves.jpeg}\\
$$\lambda \;is\; wavelength\; of\; wave;$$$$crest \;is\; peak(highest\; point) \;of\; wave;$$$$trough\; is\; dip(lowest\; point) \;of\; wave;$$
$$2\pi f \;is \; called\; angular frequency;$$

On comparing our problem with equation (3) , equation for different cases are given\\\\
$equation \;of\; sound\; wave\; when\; whistle\; is\; blown\; by$
\textit{train is}
$$y(t) = Asin( 2 \pi \times400\times t + \phi ) $$ 
\;\;\;\;\;\;\;\;\;\;\;\;\;\;\;\;\;\;\;\;for this case $f\;=\;400Hz$\\\\
$equation \;of\; sound\; wave\; observed\; by\; observer\;when\\ \;train\;is\; approaching\; observer\;$
$$y(t) = Asin( 2 \pi \times412.1212\times t + \phi ) $$ 
\;\;\;\;\;\;\;\;\;\;\;\;\;\;\;\;\;\;\;\;for this case $f\;=\;412.1212Hz$\\\\
$equation \;of\; sound\; wave\; observed\; by\; observer\;when\\ \;train\;is\; receding\; observer\;$
$$y(t) = Asin( 2 \pi \times388.5714\times t + \phi ) $$ 
\;\;\;\;\;\;\;\;\;\;\;\;\;\;\;\;\;\;\;\;for this case $f \;\;= \;\;388.5714Hz$\\
\subsection*{\textbf{Doppler Effect for Sound Waves :-}}
Doppler effect for sound wave refers to change in frequency or pitch of sound wave observed by an observer when there is a relative motion between observer and source .\\\\

    \includegraphics[width=0.89\linewidth]{figs/doppler.jpg}\\\\

\subsection*{\textbf{Derivation \;of \;Doppler :-}}
To derive Doppler , we can write equation of sound as shown
\begin{align}f = \frac{v}{\lambda}\end{align}
using equation (4) , we get
\begin{align}y(t) = Asin( 2 \pi \frac{v}{\lambda}t + \phi ) \end{align}  
$$v \;is\; speed\; of\; sound\; in\; that\; medium$$
\textbf{1.} Source is moving toward stationary Observer-\\\\
Now consider the relative motion in which source is moving towards observer , in that case effective wavelength $\lambda'$ observed by observer will be compressed ,
$$v_s \;is\; velocity\; of\; source$$
$$v_o \;is\; velocity\; of\;observer $$
$$v_s = v_s$$
$$v_o = 0$$
\begin{align}\lambda' = \lambda - v_s T\end{align}
T is time period(time taken by source wave to complete one revolution)
and effective frequency \textit{f}' observed by observer will be
\begin{align}f' = \frac{v}{\lambda'}\end{align}
using equations (6) and (7) , we get
\begin{align}f' = \frac{v}{\lambda- v_s T}\end{align}
$$f' = \frac{v f}{f(\lambda- v_s T)}$$
we know ,
\begin{align}T = \frac{1}{f}\end{align}
using equation (9)
\begin{align}f' = \frac{v f}{v- v_s }\end{align}
\bigskip\\
\textbf{2.} Source is moving away from stationary Observer-\\\\
Similarly , if source is receding from observer than $\lambda, $will be increased
$$v_s = v_s$$
$$v_o = 0$$
\begin{align}\lambda' = \lambda + v_s T\end{align}
using equations (7) and (11) , we get
\begin{align}f' = \frac{v}{\lambda+ v_s T}\end{align}
$$f' = \frac{v f}{f(\lambda+v_s T)}$$
using equation (9)
\begin{align}f' = \frac{v f}{v+ v_s }\end{align}
\newpage
\textit{Doppler effect depends on relative velocity }, so we will use this concept to prove frequencies for different cases depending on situation .\\\\
\textbf{3.} Observer is moving towards Stationary Source-\\\\
In this case , the velocity at which sound is approaching observer will increase .
$$v_s = 0$$
$$v_o = v_o$$
\begin{align}f' = \frac{v'}{\lambda'}\end{align}
\begin{align}v'= v+v_o\end{align}
$$But \;what\; about\; wavelength??$$
It's answer is , wavelength will be same  .\\

    \includegraphics[width=0.9\linewidth]{figs/wave3.png}\\\\
Sound properties only depends on situation of source and not observer .
\begin{align}\lambda' = \lambda\end{align}
using equations (15) and (16) , and substituting in equation (14) 
\begin{align}f' = \frac{v+v_o}{\lambda}\end{align}
using equation (4) , we get 
\begin{align}f' = \frac{(v+v_o) f}{v }\end{align}
\textbf{4.} Observer is moving away from Stationary Source-\\\\
In this case , the velocity at which sound is approaching observer will decrease .
$$v_s = 0$$
$$v_o = v_o$$
\begin{align}v'= v-v_o\end{align}
In this case also , wavelength will not change.
$$\lambda' = \lambda$$
using equations (19) and (16) , and substituting in equation (14) 
\begin{align}f' = \frac{v-v_o}{\lambda}\end{align}
using equation (4) , we get 
\begin{align}f' = \frac{(v-v_o) f}{v }\end{align}\\
\textbf{5.} Source and Observer are both moving towards each other-\\\\
In this case , the velocity at which sound is approaching observer will increase and wavelength will compress .
$$v_s = v_s$$
$$v_o = v_o$$
\begin{align}v'= v+v_o\end{align}
\begin{align}\lambda' = \lambda - v_s T\end{align}
using equations (22) and (23) , and substituting in equation (14) 
\begin{align}f' = \frac{v+v_o}{\lambda-v_s T}\end{align}
using equation (4) , we get 
\begin{align}f' = \frac{(v+v_o) f}{v-v_s }\end{align}
\textbf{6.} Source and Observer are both moving away from each other-\\\\
In this case , the velocity at which sound is approaching observer will decrease and wavelength will stretch .
$$v_s = v_s$$
$$v_o = v_o$$
\begin{align}v'= v-v_o\end{align}
\begin{align}\lambda' = \lambda + v_s T\end{align}
using equations (26) and (27) , and substituting in equation (14) 
\begin{align}f' = \frac{v-v_o}{\lambda+v_s T}\end{align}
using equation (4) , we get 
\begin{align}f' = \frac{(v-v_o) f}{v +v_s}\end{align}
\textbf{7.} Source is moving towards Observer and Observer moving away from Source-\\\\
In this case , the velocity at which sound is approaching observer will decrease and wavelength will compress .
$$v_s = v_s$$
$$v_o = v_o$$
\begin{align}v'= v-v_o\end{align}
\begin{align}\lambda' = \lambda - v_s T\end{align}
using equations (30) and (31) , and substituting in equation (14) 
\begin{align}f' = \frac{v-v_o}{\lambda-v_s T}\end{align}
using equation (4) , we get 
\begin{align}f' = \frac{(v-v_o) f}{v -v_s}\end{align}
\textbf{8.} Source is moving away from Observer and Observer is moving towards Source-\\\\
In this case , the velocity at which sound is approaching observer will increase and wavelength will stretch.
$$v_s = v_s$$
$$v_o = v_o$$
\begin{align}v'= v+v_o\end{align}
\begin{align}\lambda' = \lambda + v_s T\end{align}
using equations (34) and (35) , and substituting in equation (14) 
\begin{align}f' = \frac{v+v_o}{\lambda+v_s T}\end{align}
using equation (4) , we get 
\begin{align}f' = \frac{(v+v_o) f}{v +v_s}\end{align}\\\\
\textbf{9.} Both Source and Observer are stationary-\\\\
If both Source and Observer are stationary , it means 
$$v_s = 0$$
$$v_o = 0$$
also there will be no change in wavelength 
\begin{align}\lambda' = \lambda\end{align}
\begin{align}v'= v\end{align}
using equation (38) and (39), and substituting in equation (14) 
\begin{align}f' = \frac{v}{\lambda}\end{align}
using equation (4) , we get 
\begin{align}f' = f\end{align}\\\\

So , Doppler effect depends on relative velocity of Observer and Source with respect to same frame and also velocity of Sound in that medium .\\\\
On next page , we are providing a table in which various formulas of frequencies are written depending on situation .
\newpage
    \begin{table}[h]
        \input{tables/table}
    \end{table}
\end{document}
